\chapter*{Introduction}
\addcontentsline{toc}{chapter}{Introduction}
\markboth{}{Introduction}
%\epigraph{\emph{Sometimes it’s necessary to do unnecessary things.}}{Kanade Jinguuji - Best Student Council}
\epigraph{\emph{I may make you feel, but I can't make you think.}}{Jethro Tull}

Leucippus and Democritus, two Greek philosophers in the 5th century BCE, were the first recorded people to propose the idea that matter is composed of small indivisible particles called atoms.%~\cite{pullman2001atom}. 
Particle physics has come a long way since then, culminating in the 20th century with the overwhelming success of the \ac{SM}~\cite{MartinB.R.BrianRobert1997Pp} theory and the discovery of sub-atomic particles. 
We have now a better understanding of the matter that constitutes the universe than we've ever had, albeit that can be said to be true for point in the history of mankind. Perhaps what differs now is that we have a better understanding of what we know we don't know. 
Thanks to the technological advances achieved in the last century, particle physicists have been able to study the most elusive of elementary particles at energies that were only present moments after the Big Bang. 
This resulted in many discoveries being made and many questions answered. 
But there are still fundamental problems that need to be solved before we can really understand this universe we inhabit.

The \ac{SM} theory provides a framework that describes the known elementary particles and their interactions. 
Its robustness and prediction powers have been tested experimentally many times by a large plethora of different experiments, some of which have been based at the \ac{CERN} such as the \ac{LEP}, which tested the so-called electroweak sector, and the \ac{ATLAS} and \ac{CMS} experiments at \ac{LHC}, which were the first to identify the Higgs boson. 
However, the \ac{SM} is still incomplete as it fails to address problems such as in the Higgs sector or explain \textit{dark matter}. 
One of the most established extensions to the \ac{SM} proposed to address these issues is \ac{SUSY}. 
In this theory fermion-boson symmetry is introduced, giving \ac{SM} particles corresponding \ac{SUSY} partners, with a mass around the 1 \tev\ scale.
The \ac{SUSY} extension gives rise to a particle that fits the characteristics of dark matter and is able to provide a natural solution to the hierarchy problem introduced by the Higgs boson mass.

This thesis presents the work carried out over a 3.5-year \ac{PhD} degree in the search of \ac{SUSY} focusing on the direct stau-slepton (\stau) production from proton-proton collisions with fully hadronic final state and missing transfer momentum. 
The author was part of the group, within the \ac{ATLAS} collaboration, that performed this analysis using data recorded between 2015 and 2018 at centre-of-mass energies of \com$=13$ \tev\ at the \ac{LHC}. 
The resulting work has been described in a paper published in the \textit{Physical Review D} journal~\cite{PhysRevD.101.032009} in February 2020. 
The thesis will begin with Chapter~\ref{ch:theory}, which described the theoretical concepts and motivations for \ac{SUSY} searches along with their current status.
This will be followed by a description of the experimental apparatus given in Chapter~\ref{ch:detector}.
A general description of the \ac{ATLAS} trigger system alongside with the work performed by the author on the \ac{ATLAS} \ac{ID} trigger, as part of his qualification task, is presented in Chapter~\ref{ch:trigger}. 
\color{purple} The work done in the study of the performance of the \ac{ID} trigger has been summarised in a paper published in (ADD JOURNAL NAME).\color{black}
Chapter~\ref{ch:analysis} presents the analysis carried out by the author as part of the direct-stau analysis team within the \ac{ATLAS} \ac{SUSY} working group. The author had major contributions in the optimisation of the signal region definitions with the performance study done on combined \ltau\ triggers, and in the estimation of theory uncertainties for the main \ac{SM} background processes. 
The author has also had significant contribution in the development of a novel technique used for the estimation of mis-identified jets faking tau-leptons (\ftau) in the \ac{ATLAS} detector, called \textit{Universal Fake Factor} method. 
Once fully developed this method will be available to any analysis involving \ltau s, alongside with a tool that will use this method as basis to estimate the contribution of \ftau 's in any given region. A full description of the method, tool development, and most recent results is given in Chapter~\ref{ch:fake_est}.
  