\chapter*{Conclusions}
\addcontentsline{toc}{chapter}{Conclusions}
\markboth{}{Conclusions}
%\epigraph{\emph{I'm through with it... this spiral of death and killing}} {Shishido Baiken - Vagabond}
\epigraph{\emph{All in all, you're just another brick in the wall.}} {Pink Floyd}

 In this thesis the best results to date for the search of the direct production of the supersymmetric partner of the \ltau, from $pp$ collisions delivered by the \ac{LHC} with a centre-of-mass energy $\sqrt{s}=13$ \tev, are presented. 
Data collected by the \ac{ATLAS} experiment between 2015 and 2018, corresponding to 139 \infb, is used to search for the \stau\ particle that decays to two hadronically decaying \ltau\ and transverse missing energy, \met.
% and set limits on the on the parameters of the simplified electroweak supersymmetry models.
Two \acp{SR} are derived and optimised to tackle the high and low \stau\ signal masses separately. 
The author had leading involvement in the study of the performance of the tau triggers both in the inner detector, and as a full trigger chain used for the definition of the \acp{SR}. 
The \ttbar, \ttV, \Zjets, single top, and multiboson irreducible \ac{SM} backgrounds are estimated using dedicated \acp{VR} derived using \ac{MC} simulated samples. The reducible \ac{SM} backgrounds derive from the mis-identification of \ltau\ from the multijet and \Wjets\ processes and are estimated from data using the \textit{ABCD method} and a dedicated W\ac{CR}, respectively.

The theory uncertainties have been estimated by the author for the main \ac{SM} background processes, which combined with the experimental uncertainties are used by the fitting procedure as the total systematic uncertainties associated with the analysis. 
No significant deviation from the expected \ac{SM} background events is observed in the constructed \acp{VR}, indicating an accurate estimation of background contributions. 

Statistical fits have been used to extract the normalization factors used to derive the expected \ac{SM} event yield in the defined \acp{SR}. 
In the absence of any significant excess over the expected \ac{SM} background, the observed and expected numbers of events are used to set 95\% \ac{CL} exclusion limits on the parameters of the simplified electroweak supersymmetry models.
Stau masses from 120 \gev\ to 390 \gev\ are excluded for a massless lightest neutralino, for this scenario. 
Limits on the simplified model production of left-handed-staus (\stauL), with masses between 155 \gev\ and 310 \gev\ for a massless lightest neutralino, have also been set .

A novel technique, named \textit{Universal Fake Factor} method, currently under development for the estimation of mis-identified \htau\ objects in any arbitrary \ac{SR}. This is described in detail in Chapter~\ref{ch:fake_est}. 
%The study and derivation of the parton track widths used for the tool fitting procedure has been conducted primarily by the author of this paper. 
Once fully developed this method will be able to produce \textit{transfer factors} that can be used for the estimation of the main sources of reducible backgrounds for any analysis involving \ltau s. 



